%%%%%%%%%%%%%%%%%%%%%%%%%%%%%%%%%%%%%%%%%
% "ModernCV" CV and Cover Letter
% LaTeX Template
% Version 1.2 (25/3/16)
%
% This template has been downloaded from:
% http://www.LaTeXTemplates.com
%
% Original author:
% Xavier Danaux (xdanaux@gmail.com) with modifications by:
% Vel (vel@latextemplates.com)
%
% License:
% CC BY-NC-SA 3.0 (http://creativecommons.org/licenses/by-nc-sa/3.0/)
%
% Important note:
% This template requires the moderncv.cls and .sty files to be in the same 
% directory as this .tex file. These files provide the resume style and themes 
% used for structuring the document.
%
%%%%%%%%%%%%%%%%%%%%%%%%%%%%%%%%%%%%%%%%%

%----------------------------------------------------------------------------------------
%	PACKAGES AND OTHER DOCUMENT CONFIGURATIONS
%----------------------------------------------------------------------------------------

\documentclass[10pt,a4paper,roman]{moderncv} % Font sizes: 10, 11, or 12; paper sizes: a4paper, letterpaper, a5paper, legalpaper, executivepaper or landscape; font families: sans or roman

\moderncvstyle{casual} % CV theme - options include: 'casual' (default), 'classic', 'oldstyle' and 'banking'
\moderncvcolor{black} % CV color - options include: 'blue' (default), 'orange', 'green', 'red', 'purple', 'grey' and 'black'

\usepackage{lipsum} % Used for inserting dummy 'Lorem ipsum' text into the template
\usepackage{color}
\usepackage[scale=0.75]{geometry} % Reduce document margins
%\usepackage[top=1.5in, bottom=1.5in, left=1.5in, right=1.5in, footskip=0.66in]{geometry}
\setlength{\hintscolumnwidth}{3.3cm} % Uncomment to change the width of the dates column
%\setlength{\makecvtitlenamewidth}{7cm} % For the 'classic' style, uncomment to adjust the width of the space allocated to your name
\usepackage[T1]{fontenc}
\usepackage[utf8]{inputenc}
\usepackage[english, norsk]{babel}
\usepackage{hyperref}
\usepackage{xcolor}

%----------------------------------------------------------------------------------------
%	NAME AND CONTACT INFORMATION SECTION
%----------------------------------------------------------------------------------------

\firstname{F.Ceren} % Your first name
\familyname{Ay} % Your last name

% All information in this block is optional, comment out any lines you don't need
\title{Curriculum Vitae}
\address{Norwegian School of Economics-Department of Economics}{ Helleveien 30, 5045,Bergen, Norway}
%\mobile{+41 (0) 76 534 08 17}
\phone{+47 92 25 66 45}
\email{Fehime.ay@nhh.no}
%\homepage{https://wwz.unibas.ch/personen/profil/person/roth/}{https://wwz.unibas.ch/personen/profil/person/roth/} % The first argument is the url for the clickable link, the second argument is the url displayed in the template - this allows special characters to be displayed such as the tilde in this example
%\extrainfo{additional information}
%\photo[70pt][0.4pt]{pictures/picture} % The first bracket is the picture height, the second is the thickness of the frame around the picture (0pt for no frame)
%\quote{"A witty and playful quotation" - John Smith}

%----------------------------------------------------------------------------------------

\begin{document}
%\color{gray}
%----------------------------------------------------------------------------------------
%	COVER LETTER
%----------------------------------------------------------------------------------------

% To remove the cover letter, comment out this entire block

%\clearpage
%
%\recipient{HR Department}{Corporation\\123 Pleasant Lane\\12345 City, State} % Letter recipient
%\date{\today} % Letter date
%\opening{Dear Sir or Madam,} % Opening greeting
%\closing{Sincerely yours,} % Closing phrase
%\enclosure[Attached]{curriculum vit\ae{}} % List of enclosed documents
%
%\makelettertitle % Print letter title
%
%\lipsum[1-3] % Dummy text
%
%\makeletterclosing % Print letter signature
%
%\newpage

%----------------------------------------------------------------------------------------
%	CURRICULUM VITAE
%----------------------------------------------------------------------------------------

\makecvtitle % Print the CV title
\section{Personal Information}
%\cvitem{Affiliation}{Bocconi University - Department of Accounting}
%\cvitem{E-mail}{mert.erinc@phd.unibocconi.it}
\cvitem{Citizenship}{Turkish}
\cvitem{Date of Birth}{11 April 1991}
\cvitem{Web Page:}{\href{https://cerenay.github.io}{\color{blue}cerenay.github.io}}
%----------------------------------------------------------------------------------------
%	EDUCATION SECTION
%----------------------------------------------------------------------------------------
\section{Research Interests}
\cvitem{}{Behavioral Economics, Experimental Economics, Self and Social Image Effects,
Information Preferences, Social Decisions}
%----------------------------------------------------------------------------------------
%	Employment
%----------------------------------------------------------------------------------------

\section{Employment}

\cvitem{10/2020- today}{Part Time Lecturer at Norwegian School of Economics(NHH), Department
of Economics}
\cvitem{08/2016- today}{PhD Research Scholar at Norwegian School of Economics(NHH), Department
of Economics, FAIR - The Choice Lab}

\cvitem{10/2014-08/2016}{Research Assistant at Istanbul University- Faculty of Economics}


  % Arguments not required can be left empty
%----------------------------------------------------------------------------------------
%	EDUCATION SECTION
%----------------------------------------------------------------------------------------


\section{Education}

\cvitem{08/2016 - 12/2020(Expected)}{ PhD in Economics at NHH - Department of Economics, FAIR - The Choice Lab
supervised by Erik \O iolf S\o rensen
\textit{(External supervisor: Pedro Dal B\'{o})}} 

\cvitem{01/2019 -- 06/2019}{Visiting Researcher - Brown University, Department of Economics, USA \newline 
\textit{Host: Louis Putterman}}
%\cvitem{Research Visit - Brown University (January 2019 - June 2019)}}

\cvitem{09/2014 - 07/2016}{Master of Arts in Economics, Galatasaray University \newline
	\textit{ Thesis: "A Voting Experiment on Promises and Fairness Perceptions"}} 
	
\cvitem{09/2009 - 01/2014}{ Bachelor of Arts in Economics, Hacettepe University\newline 
    \textit{Thesis:  "Game Theoretical Approach to Tax Evasion and Vickrey-Clark-Groves Mechanism"}}

%\cvitem{2011 -- 2012}{Exchange Program - Antwerp University, Belgium}

%\cvitem{2009 -- 2013}{Bachelor of Arts in Business Administration, Marmara University, Turkey \newline
%	\textit{Major: `` Accounting and Finance'' }}

\newpage
\section{Research}
\cvitem{Work in Progress}{\textbf{Information Avoidance and Image Concerns in Reciprocal Decisions}}
\cvitem{}{\textit{This paper provides evidence on strategic use of information to dampen moral pressure when making decisions. Strategic use of information refers to people actively avoiding or acquiring information rather than exogenously given setting. In a manipulated trust game two aspects of information about the consequences of the decision is varied: time of the information and exogenous information / ignorance. Results show that when people make decisions under exogenous ignorance, reciprocity drops substantially compared to the settings with exogenous information. When people are given the chance to choose whether to know or not to know the consequences of their decisions, there is a clear relation between more selfish decisions and information avoidance. Contrary to previous literature, this pattern is observed both before making the decision (ex-ante) and after the decision is already made (ex-post). This provides evidence on the psychological utility of information by signalling the prosociality to one's self both in ex-ante and ex-post perspective. This study provides an insightful view on information preferences in a relatively complex social settings by showing how information preferences are used to make more selfish decisions.}}
%\cvitem{Progress:}{\textit{Currently writing - Data collection is completed in October 2018}}


\cvitem{Work in Progress}{\textbf{Strategic Curiosity - joint with Katrine B. Nødvedt (NHH) and Joel Berge (NHH)}}
\cvitem{}{\textit{This study provides experimental evidence on a novel phenomenon in information preferences: people strategically collect additional non-instrumental information to justify morally questionable decisions. We conduct a virtual dice-rolling experiment in which we vary the extent to which participants can collect additional information before reporting as well as the content of information. We document a tendency to systematically collect more information \textit{-be more curious-} when tempted to misreport. Curiosity is positively correlated with the size of the lie. Interestingly, neither curiosity nor dishonesty respond to the content of information as people roll the dice and misreport to the same extent even when information is irrelevant to the decision. Our study provides new insights into how individuals actively shape their information environment in pursuit of self-interest. }}
%\cvitem{Progress:}{\textit{Currently writing - Data collection is completed in August 2019}}

\cvitem{Work in Progress}{\textbf{Reasoning Avoidance - joint with Hallgeir Sj\aa stad (NHH) and Steven Sloman (Brown University)}}
\cvitem{}{\textit{This research project aims to investigate individuals' behavior when they are asked to explain a spurious relation between a main event and a side effect which can potentially challenge existing beliefs on the main event in an experimental setup. The main contribution of the present research is showing how causal relations can be misconceived by individuals when they challenge the beliefs and preferences about the main event -policy.}}

%\cvitem{Work in Progress}{\textbf{Self-Image Considerations in the Provision of Helpful Feedback - joint with Stefan Mei{\ss}ner (NHH)}}
%\cvitem{}{\textit{The goal of the project is to see how individuals view others' self-image concerns and information avoidance decision and is that view affected by the material and ego utility/disutility of the information}}

%\cvitem{Work in Progress}{\textbf{Perform Better Under Self-selected Incentives? A Study on Performance Effects of Self-selection of Incentive Choice -joint with Xiaogeng Xu (Hanken School of Economics)}}
%\cvitem{}{\textit{This study aims to explore the role of self-selection of incentive in performance with a lab experiment where participants perform a task with real efforts. Piece-rate payment and incentives for performance are either selected by the workers themselves, or exogenously imposed.}}

\cvitem{Work in Progress}{\textbf{An International Collaboration on Social and Moral Psychology of Covid-19 - joint with Jay van Bavel and others}}
\cvitem{}{\textit{The goal of this collaboration is to bring together scholars from around the globe to examine psychological factors underlying the attitudes and behavioral intentions related to Covid-19. To date, data from over 44.000 citizens in 67 countries are collected.} 
\href{https://psyarxiv.com/ydt95}{\color{blue}Working Paper} and \href{https://icsmp-covid19.netlify.app}{\color{blue}Project website}}

\newpage
\section{Presentations at Conferences and Seminars}
\cvitem{November 2019}{Joint PhD Workshop NHH and University of Bergen (NHH Bergen, Norway)}
\cvitem{September 2019}{PhD Course with Lise Vesterlund - Identification Through Experiments (NHH, Bergen Norway)}
\cvitem{April 2019}{Brown Bag Seminar (Brown University, Sloman Lab, Providence, USA)}
\cvitem{March 2019}{Spring School in Behavioral Economics - Poster Session (Rady School of Management, UC San Diego, USA)}
\cvitem{January 2019}{Brown Bag Seminar (Brown University, Department of Economics, Providence, USA)}
\cvitem{November 2018}{Joint PhD Workshop NHH and University of Bergen (NHH Bergen, Norway)}
\cvitem{June 2018}{FAIR Inaugural Conference - Poster Session (NHH, Bergen, Norway)}
\cvitem{June 2016}{7th International Conference of the French Association of Experimental Economics (Essec Business School, Cergy, France)}
%----------------------------------------------------------------------------------------
%	WORK EXPERIENCE SECTION
%----------------------------------------------------------------------------------------
%\section{Awards and Scholarships}
%\cvitem{2016 - present}{PhD Scholarship, Bocconi University}
\section{Workshop Attendance}
\cvitem{July 2019}{2nd briq Summer School in Behavioral Economics, University of Bonn}
\cvitem{March 2019}{Spring School in Behavioral Economics, Rady School of Management, UC San Diego, US}
\section{Teaching Assistance}
\cvitem{Spring 2020}{Behavioral Economics, NHH}
\cvitem{Fall 2019}{Econometric Techniques, NHH}
\cvitem{Fall 2018}{Human Capital, Mobility and Diversity in Firms, NHH}
\cvitem{Fall 2018}{Econometric Techniques, NHH}
\cvitem{Spring 2018}{Long Term Macroeconomic Analysis, NHH}
\cvitem{Fall 2017}{Ethics and Diversity in Firms, NHH}

\section{Languages and Skills}
\cvitem{Languages}{Turkish (native), English (fluent), Norwegian(basic)}
\cvitem{Software}{ Latex, Mathematica, oTree, Python, R, Stata, zTree}

\section{Academic References}

\cvitem{Prof. Erik \O iolf
S\o rensen}{NHH - Department of Economics, FAIR-The Choice Lab,   \hspace{10em} \strut\hspace{0.5em}
Helleveien 30, 5045, Bergen, Norway \hspace{15em} \strut    \hspace{5em}
erik.sorensen@nhh.no}

\cvitem{Prof. Bertil
Tungodden}{NHH - Department of Economics, FAIR-The Choice Lab,   \hspace{10em} \strut    \hspace{0.5em}
Helleveien 30, 5045, Bergen, Norway \hspace{15em} \strut    \hspace{5em}
bertil.tungodden@nhh.no}

%\cvitem{Prof. Alexander Cappelen}{NHH - %Department of Economics, FAIR-The Choice Lab,   %\hspace{10em} \strut    \hspace{0.5em}
%Helleveien 30, 5045, Bergen, Norway %\hspace{15em} \strut    \hspace{5em}
%alexander.cappelen@nhh.no}

\cvitem{Prof. Louis Putterman}{Brown University - Department of Economics,   \hspace{12em} \strut    \hspace{0.5em}
70 Waterman Street, Providence, RI 02912, USA \hspace{8em} \strut    \hspace{0em}
louis\_putterman@brown.edu}

\cvitem{Prof. Pedro 
Dal B\'{o}}{Brown University - Department of Economics,   \hspace{13em} \strut    \hspace{10em}
70 Waterman Street, Providence, RI 02912, USA \hspace{8em} \strut    \hspace{0em}
pedro\_dal\_bo@brown.edu}


%\cvitem{Assoc. Prof. Bilge {\"O}zt{\"u}rk G{\"o}ktuna}{Galatasaray University, Department of Economics,
%\hspace{12em} \strut    \hspace{0.5em}
%Ortakoy Mh., Ciragan Cd. No:36, 34349, Besiktas,Istanbul  \hspace{8em} \strut    
%    \hspace{0em}
%bozturk@gsu.edu.tr}

%\cvitem{Assoc. Prof. Mehmet Yi\u{g}it G{\"u}rdal} {Bogazici University, Department of Economics,
%\hspace{12em} \strut    \hspace{0.5em}
%Bebek Mh., 34342, Besiktas, Istanbul  \hspace{15em} 
%\strut  \hspace{5em}
%mehmet.gurdal@boun.edu.tr}



%----------------------------------------------------------------------------------------
%	WORK EXPERIENCE SECTION
%----------------------------------------------------------------------------------------
%
%\section{Experience}
%
%\subsection{Vocational}
%
%\cventry{2012--Present}{1\textsuperscript{st} Year Analyst}{\textsc{Lehman Brothers}}{Los Angeles}{}{Developed spreadsheets for risk analysis on exotic derivatives on a wide array of commodities (ags, oils, precious and base metals), managed blotter and secondary trades on structured notes, liaised with Middle Office, Sales and Structuring for bookkeeping.
%\newline{}\newline{}
%Detailed achievements:
%\begin{itemize}
%\item Learned how to make amazing coffee
%\item Finally determined the reason for \textsc{PC LOAD LETTER}:
%\begin{itemize}
%\item Paper jam
%\item Software issues:
%\begin{itemize}
%\item Word not sending the correct data to printer
%\item Windows trying to print in letter format
%\end{itemize}
%\item Coffee spilled inside printer
%\end{itemize}
%\item Broke the office record for number of kitten pictures in cubicle
%\end{itemize}}
%
%%------------------------------------------------
%
%\cventry{2010--2011}{Summer Intern}{\textsc{Lehman Brothers}}{Los Angeles}{}{Rated "truly distinctive" for Analytical Skills and Teamwork.}
%
%%------------------------------------------------
%
%\subsection{Miscellaneous}
%
%\cventry{2008--2009}{Computer Repair Specialist}{Buy More}{Burbank}{}{Worked in the Nerd Herd and helped to solve computer problems by asking customers to turn their computers off and on again.}


%----------------------------------------------------------------------------------------
%	COMPUTER SKILLS SECTION
%----------------------------------------------------------------------------------------
%
%\section{Computer skills}
%
%\cvitem{Basic}{\textsc{java}, Adobe Illustrator}
%\cvitem{Intermediate}{\textsc{python}, \textsc{html}, \LaTeX, OpenOffice, Linux, Microsoft Windows}
%\cvitem{Advanced}{Computer Hardware and Support}
%
%%----------------------------------------------------------------------------------------
%%	COMMUNICATION SKILLS SECTION
%%----------------------------------------------------------------------------------------
%
%\section{Communication Skills}
%
%\cvitem{2010}{Oral Presentation at the California Business Conference}
%\cvitem{2009}{Poster at the Annual Business Conference in Oregon}
%
%%----------------------------------------------------------------------------------------
%%	LANGUAGES SECTION
%%----------------------------------------------------------------------------------------
%
%\section{Languages}
%
%\cvitemwithcomment{English}{Mothertongue}{}
%\cvitemwithcomment{Spanish}{Intermediate}{Conversationally fluent}
%\cvitemwithcomment{Dutch}{Basic}{Basic words and phrases only}
%
%%----------------------------------------------------------------------------------------
%%	INTERESTS SECTION
%%----------------------------------------------------------------------------------------
%
%\section{Interests}
%
%\renewcommand{\listitemsymbol}{-~} % Changes the symbol used for lists
%
%\cvlistdoubleitem{Piano}{Chess}
%\cvlistdoubleitem{Cooking}{Dancing}
%\cvlistitem{Running}
%
%%----------------------------------------------------------------------------------------

\end{document}